\chapter{Reconstrucción e identificación de objetos físicos}
% \addcontentsline{toc}{chapter}{Reconstrucción e identificación de objetos físicos}
\chaptermark{Reconstrucción e identificación de objetos físicos}


La reconstrucción de fotones y electrones en ATLAS se basa en las deposiciones locales de energía halladas en el ECAL, y la distinción entre unos y otros se   realiza mediante la información de las trazas reconstruidas en el ID. A su vez se aplican una serie de criterios de identificación y aislamiento, que permiten discriminarlos de falsos candidatos, o de procesos secundarios que los producen.

\section{Reconstrucción de electrones y fotones}

La  reconstrucción de fotones y electrones en ATLAS se basa en un algoritmo de clusterización \cite{Lampl:1099735} que busca deposiciones locales de energía en el calorímetro dentro de una ventana rectangular en el espacio ($\eta$, $\phi$) de tamaño fijo (Sliding Window clusterization (SW)). La posición de la ventana se  justa maximizando la energía transversa de todas las celdas contenidas. El tamaño óptimo de la ventana depende del tipo de partícula (más ancha para los electrones) a reconstruir y de la región del calorímetro (mas ancha en la región endcap). 

Aquellos clusters electromagnéticos asociados con una traza reconstruída con $p_{T} > 0.5 \egev$, son clasificados como electrones. La definición para fotones es un poco más complicada ya que estos pueden convertir en un par $e^{+}e^{-}$ en el sector anterior al calorímetro. Los fotones convertidos están caracterizados por la presencia de al menos una traza asociada proveniente de un vértice reconstruido en el ID. La probabilidad de conversión varía entre un 40\% y un 80\% dependiendo de $\eta$, aunque solo aquellas que ocurren antes del TRT son eficientemente reconstruidas. 

Si no hay ninguna traza asociada a un dado cluster, este es clasificado como un fotón no convertido. Aquellos clusters asociados con trazas, que provienen de un vértice reconstruído en el ID, es clasificado como un fotón convertido. Además, para incrementar la eficiencia de reconstrucción de estos últimos, se consideran también aquellos casos donde solo una traza fue reconstruida, siempre que esta no posea ningún impacto en el B-layer.

\section{Identificación de electrons y fotones}

La identificación de fotones y electrones se lleva a cabo mediante una serie de cortes rectangulares en un conjunto de variables que describen la forma y la estructura de las lluvias electromagnéticas según se propagan en el detector. Estas variables incluyen información de los calorímetros y, para el caso de electrones o fotones convertidos, del detector de trazas.

Para los electrones se definien tres conjuntos de cortes dependiendo de la rigurosidad de los mismos: \textit{loose}, \textit{medium}, \textit{tight}. Para los fotones solo se utiliza \textit{loose} y \textit{tihgt}. Los cortes de cada conjunto han sido optimizados para asegurar una alta eficiencia de identificación de electrones/fotones aislados y de rechazo de fondo.

La definición de las variables que defninen a los cortes se detallan a continuación. La defeinición de los cortes se muestra en la Tabla \ref{lmttable}.


\begin{itemize}

	\item \textbf{Fuga hadrónica}

		Es la energía transversa depositada en el calorímetro hadrónico, normalizada a la energía transversa del cluster electromagnético.

		\begin{equation}
		R_{had_{(1)}}=\frac{E_{T}^{had}}{E_{T}}
		\end{equation}

		En la región de transición barrel-endcap del HCAL, se utiliza el depósito de energía en todo el calorímetro hadrónico para minimizar los efectos de la degradación de resolución ($R_{had}$). En el resto del detector, se mide sólo la energía hadrónica depositada en la primera capa del HCAL ($R_{had_{(1)}}$).

	\item  Perfil lateral de energía en $\eta$

		\begin{equation}
		R_{\eta}=\frac{E_{3\times 7}^{S2}}{E_{7\times 7}^{S2}}
		\end{equation}

 		donde $E_{i\times j}^{S2}$ es la suma de las celdas en la segunda capa del calorímetro electromagnético contenidas en una ventana $i\times j$.

	\item  Perfil lateral de energía en $\phi$

		\begin{equation}
		R_{\phi}=\frac{E_{3\times 3}^{S2}}{E_{3\times 7}^{S2}}
		\end{equation}

 		donde $E_{i\times j}^{S2}$ es la suma de las celdas en la segunda capa del calorímetro electromagnético contenidas en una ventana $i\times j$.

	\item  RMS del perfil lateral de energía en $\eta$


		\begin{equation}
		w_{\eta_{2}}=\sqrt{\frac{\sum E_{i}\eta_{i}^{2}}{\sum E_{i}}- \left(\frac{\sum E_{i}\eta_{i}}{\sum E_{i}}\right)^{2}}
		\end{equation}

		mide el ancho lateral de las lluvias electromagnéticas, donde $E_{i}$ es la energía de la i-ésima celda del calorímetro electromagnético contenida en una ventana de $3\times 5$ celdas en $\eta \times \phi$.

	\item  Perfil lateral de energía en $\eta$

		\begin{equation}
		F_{side}=\frac{E(\pm 3)-E(\pm 1)}{E(\pm 1)}
		\end{equation}

		mide la contención lateral de la cascada electromagnética a lo largo de $\eta$. $E(\pm n)$ es la energía en las $\pm n$ celdas alrededor de aquella con la deposición máxima.

	\item  RMS del perfil lateral de energía en $\eta$ (3 \textit{strips})

		\begin{equation}
		w_{s,3}=\sqrt{\frac{\sum E_{i}(i-i_{max})^{2}}{\sum E_{i}}}
		\end{equation}

		mide el ancho de la lluvia electromagnética a lo largo de $\eta$ en la primera capa del calorímetro electromagnético usando solo la banda con mayor deposición de energía ($E_{i_{max}}$) y sus vecinas inmediatas.

	\item  RMS del perfil lateral de energía en $\eta$ (total)

		$w_{s,tot}$ está definida de la misma forma que $w_{s,3}$, pero utiliza todas las bandas de la primera capa del calorímetro electromagnético en una ventana $\Delta\eta\times\Delta\phi = 0.0625 \times 0.2$, que corresponde aproximadamente a $20\times 2$ bandas en $\eta \times \phi$.

	\item  Diferencia al segundo máximo

		\begin{equation}
		\Delta E=[E_{2^{nd} max}^{S1} - E_{min}^{S1}]
		\end{equation}

		es la diferencia entre la energía de la banda con la segunda energía más grande $E_{2^{nd} max}^{S1}$ , y la mínima energía $E_{min}^{S1}$ entre la anterior y la celda con la máxima deposición. En caso de no haber segundo máximo se fija $\Delta E = 0$.

	\item  Asimetría de los dos máximos locales en $\eta$

		\begin{equation}
		\Delta E_{ratio}=\frac{E_{1^{st} max}^{S1} - E_{2^{nd} max}^{S1}}{E_{1^{st} max}^{S1} + E_{2^{nd} max}^{S1}}
		\end{equation}
 
 		mide la diferencia relativa entre las energías de las dos celdas con máxima deposición. En caso de no haber segundo máximo se fija $E_{ratio} = 1$.


\end{itemize}

\renewcommand{\arraystretch}{1.3}
\begin{table}	
\centering
\begin{threeparttable}
\caption{Definición de las diferentes variables usadas para la selección \textit{loose} (L), \textit{medium} (M) y \textit{tight} (T) de fotones y electrones.}
\begin{tabular}{ r p{8cm} c | c c | c c c }

	\hline

	\multirow{2}{*}{Categoría} & \multirow{2}{*}{Descripción} & \multirow{2}{*}{Nombre} & \multicolumn{2}{ c |}{$\gamma$} & \multicolumn{3}{ c }{e} \\

		&	&	& L & T & L & M & T \\

	\hline

	Aceptancia & $|\eta| < 2.37$, excluyendo $1.37 < |\eta| < 1.52$  & - & $\times$ & $\checkmark$ & $\times$ & $\checkmark$ & $\checkmark$ \\

	Fuga hadrónica & Cociente entre $E_{T}$ en la primera capa del calorímetro hadrónico y $E_{T}$ del \textit{cluster} electromagnético & $R_{had_{1}}$ & $\checkmark$ & $\checkmark$ & $\checkmark$ & $\checkmark$ & $\checkmark$ \\

		& Cociente entre $E_{T}$ en todo el calorímetro hadrónico y $E_{T}$ del \textit{cluster} electromagnético $(|\eta| \le 0.8$ y $|\eta| \ge 1.37)$ & $R_{had}$ & $\checkmark$ & $\checkmark$ & $\checkmark$ & $\checkmark$ & $\checkmark$ \\

	ECAL (2$^{da}$ capa) & Cociente entre la suma de las energías de las 3 $\times$ 7 celdas y la suma de 5 $\times$ 7 celdas, ambas en torno al centro del \textit{cluster} & $R_{\eta}$ & $\checkmark$ & $\checkmark$ & $\checkmark$ & $\checkmark$ & $\checkmark$ \\

		& Ancho lateral de la lluvia en dirección de $\eta$ & $w_{\eta_{2}}$ & $\checkmark$ & $\checkmark$ & $\checkmark$ & $\checkmark$ & $\checkmark$ \\

		& Cociente entre la suma de las energías de las 3 $\times$ 3 celdas y la suma de 3 $\times$ 7 celdas, ambas en torno al centro del \textit{cluster} & $R_{\phi}$ & $\times$ & $\checkmark$ & $\checkmark$ & $\checkmark$ & $\checkmark$ \\

	ECAL (1$^{ra}$ capa) & Ancho lateral de la lluvia en 3 \textit{strips} alrededor del máximo & $w_{s,3}$ & $\times$ & $\checkmark$ & $\times$ & $\checkmark$ & $\checkmark$ \\

		& Ancho lateral total de la lluvia & $w_{s,tot}$ & $\times$ & $\checkmark$ & $\times$ & $\checkmark$ & $\checkmark$ \\

		& Fracción de energía fuera de las 3 \textit{strips} centrales pero dentro de las 7 & $F_{side}$ & $\times$ & $\checkmark$ & $\times$ & $\checkmark$ & $\checkmark$ \\

		& Diferencia entre la energía de la \textit{strip} con el segundo mayor depósito y la menor energía entre los dos primeros máximos locales & $\Delta E$ & $\times$ & $\checkmark$ & $\times$ & $\checkmark$ & $\checkmark$ \\

		& Asimetría entre el primer y segundo máximo & $E_{ratio}$ & $\times$ & $\checkmark$ & $\times$ & $\checkmark$ & $\checkmark$ \\

	ID & Impactos en el Pixel $\ge 1$ y en el SCT $\ge 7$ & - & $\times$ & $\times$ & $\times$ & $\checkmark$ & $\checkmark$ \\

		& Parámetro de impacto $\le 1$ mm & - & $\times$ & $\times$ & $\times$ & $\checkmark$ & $\checkmark$ \\

	ECAL+ID & $\Delta\eta$, $\Delta\phi$ entre la traza extrapolada al calorímetro y el \textit{cluster} & $\Delta\eta$, $\Delta\phi$ & $\times$ & $\times$ & $\times$ & $\checkmark$ & $\checkmark$ \\

	Fuga hadrónica & Cociente entre la energía del \textit{cluster} y el impulso de la traza & $E/p$ & $\times$ & $\times$ & $\times$ & $\checkmark$ & $\checkmark$ \\

	TRT & Impactos en el TRT & - & $\times$ & $\times$ & $\times$ & $\checkmark$ & $\checkmark$ \\

		& Fracción de impactos de alto umbral en el TRT & - & $\times$ & $\times$ & $\times$ & $\checkmark$ & $\checkmark$ \\

	\hline

\end{tabular}
\end{threeparttable}
\label{lmttable}
\end{table}
\renewcommand{\arraystretch}{1}




\section{Criterios de aislamiento}





\section{Energía faltante}

Una variable de suma importancia en la reconstrucción de objetos, es la energía transversa faltante ($E_{T}^{miss}$). Como se mencionó anteriormente, el momento transverso se debe conservar en todo el proceso, y cualquier desequilibrio en el mismo se lo asocia a $E_{T}^{miss}$. Puede indicar la presencia de partículas indetectables como neutrinos o nuevas partículas estables, o que interactúan débilmente con la materia. L

El momento transverso faltante ($p_{T}^{miss}$) se define como el valor negativo de la suma del momento transverso de todas las partículas detectadas, y su magnitud es lo que se denomina energía transversa faltante. Esta es calculada con un algoritmo basado en objetos \cite{Khoo:2012749}. El algoritmo utiliza los objetos físicos construidos y calibrados descriptos en las secciones anteriores. Los depósitos de energía en el calorímetro (topo-clusters) son asociados a los objetos de alto $p_{T}$ en el siguiente orden: electrones, fotones, jets y muones. Los depósitos que no están asociados a ningún objeto son incluidos en el término soft. El momento transverso es calculado entonces como:

\begin{equation}
p_{T}^{miss}=\left(p_{T}^{miss}\right)^{e} + \left(p_{T}^{miss}\right)^{\gamma} + \left(p_{T}^{miss}\right)^{jet} +\left(p_{T}^{miss}\right)^{\mu} + \left(p_{T}^{miss}\right)^{soft}
\end{equation}

(para discutir nomenclatura)

donde cada término es calculado como el negativo de la suma de los objetos reconstruidos y calibrados, más el térmio soft. La contribución de los taus, se incluye en el término de jets o en el soft, debido a que los mismos decaen hadrónicamente. Finalmente la energía transversa faltante se define como:

\begin{equation}
E_{T}^{miss}=|p_{T}^{miss}|
\end{equation}

\normalsize