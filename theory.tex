\chapter{Modelo Estándar y Supersimetría}
% \addcontentsline{toc}{chapter}{Modelo Estándar y Supersimetría}
\chaptermark{Modelo Estándar y Supersimetría}


El Modelo Estándar (SM, por sus siglas en inglés) es la teoría que describe a las partículas elementales y a sus interacciones. Este modelo fue introducido por Glashow, Salam y Weinberg en la década de los 70 \cite{Glashow:1961tr,PhysRevLett.19.1264} (falta Salam y cambiar estilo de citas). Está basado en teorías cuánticas de campo, y sus predicciones, cuantitativas y cualitativas, han sido verificadas experimentalmente con gran precisión.

Una de las extensiones del SM mejor motivada desde el punto de vista teórico es la Supersimetría, ya que resuelve algunas de las limitaciones del mismo. En particular, provee una solución al problema de jerarquía, proporciona candidatos para la materia oscura, permite la unificación de las fuerzas del SM, y hasta propone una conexión entre estas y la gravedad. Es por este motivo, que la Supersimetría, se ha vuelto uno de los objetivos en la búsquda de nueva física de los últimos años.

\section{Modelo Estándar}
 
Según el SM las partículas se clasifican en dos grandes grupos: fermiones y bosones. Los fermiones son los que componen la materia ordinaria y se caracterizan por obedecer la estadística de Fermi-Dirac y tener spin semientero. Estos se clasifican en leptones y quarks, según si experimentan o no la interacción fuerte, siendo los últimos los que pueden interactuar.  

Existen 6 tipos (o sabores) de leptones que se clasifican en tres generaciones. Cada generación se forma a partir de un leptón masivo y cargado y otro no masivo y neutro. Así se tienen el electrón ($e^{-}$) con su correspondiente neutrino ($\nu_{e}$), y el muón ($\mu^{-}$) y el tau ($\tau^{-}$) con sus neutrinos asociados ($\nu_{\mu}$ y $\nu_{\tau}$ ).

De la misma forma, existen 6 tipos de quarks: up (u), down (d), charm (c), strange (s), top (t) y bottom (b). A diferencia de los leptones, lo quarks tienen carga de color, que les permite interactuar mediante la fuerza fuerte. Los quarks solo se manifiestan en estados ligados, denominados hadrones, fenómeno conocido
como confinamiento de quarks. Existen dos tipos de hadrones en la naturaleza: los bariones ($qqq$) y los mesones ($q\hat{q}$).

Los fermiones se pueden encontrar en dos estados de helicidad, izquierda y derecha, salvo los neutrino que solamente existen en estados de helicidad izquierda. 

Las dos últimas generaciones de fermiones son inestables, por lo que decaen a las de la primera generación. Es por esto que la materia ordinaria está compuesta por fermiones de la primera generación. ( no se donde meter esto )

Asi como los fermiones están asociados a la materia, los bosones están asociados a los portadores de las interacciones. Los mismos se carcterizan por obedecer la estadística Bose-Einstein y por tener spin entero. Existen cuatro tipos de interacciones fundamentales. La electromagnética, que afecta a las partículas con carga eléctrica, cuyo bosón asociado es el fotón. La débil, que actúa tanto en los quarks como en los leptones, asociada a los bosones $W^{\pm}$ y $Z^{0}$. La interacción fuerte, que actúa en las partículas con carga de color, cuyo portador son los gluones. Finalmente, la cuarta interacción es la gravitatoria. La misma no está descripta por el SM, pero supone que debería actuar sobre todas las partículas del SM y su bosón asociado sería el gravitón.

Todas las partículas anteriormente mencionadas, tienen asociadas una antipartícula con la misma masa, carga y varios de sus números cuánticos opuestos (isospin, charmness, strangeness, topness, baryon, etc.). 

El SM se construye formalmente como una teoría de gauge no abeliana, imponiendo invarianza de gauge local sobre campos cuantificados que describen las partículas fundamentales, dando lugar a los campos de gauge que descibren las interacciones. Su grupo de simetria es:

\begin{equation}
\mathcal{G}_{SM}=SU(3)_{C}\times SU(2)_{L}\times U(1)_{Y}
\end{equation}


donde Y (la hipercarga), L (la helicidad izquierda) y C (la carga de color) representan las cantidades conservadas del grupo de simetría. El subgrupo $SU(2)_{L}\times U(1)_{Y}$ representa el sector electrodébil (QED + interacción débil) y el subrgupo $SU(3)_{C}$ incluye la cromodinámica cuántica (QCD).


En el SM las partículas adquieren su masa mediante el mecanismo de Higgs \cite{PhysRevLett.13.508}, a partir de la ruptura espontanea de la simetria electrodébil:

\begin{equation}
\mathcal{G}_{SM}\rightarrow SU(3)_{C}\times U(1)_{Q}
\end{equation}

produciendo los bosones masivos $W^{\pm}$ y $Z^{0}$. Como consecuencia, es necesario incluir en el lagrangiano un nuevo campo escalar, dando lugar a un nuevo bosón masivo de spin 0, llamado bosón de Higgs. El mismo fue descubierto en el año 2012 por las colaboraciones ATLAS y CMS \cite{Aad:2012tfa, Chatrchyan:2012xdj}. La masa del Higgs se midió con un valor de $125.09 \pm 0.21 (stat) \pm 0.11 (syst) \egev$ \cite{Aad:2015zhl}. Así como los bosones de gauge adquieren su masa mediante este mecanismo, es posible también generar la masa de los fermiones mediante su interacción con el Higgs, completando de esta forma el espectro de masas del SM.

La Tabla \ref{smparticles} resume las propiedades de las particualas mencionadas.

\renewcommand{\arraystretch}{1.3}
\begin{table}	
\centering
\begin{threeparttable}
\caption{Partículas elementales del SM.}
\begin{tabular}{ l | c  c  c | c c }

	\hline

		& \multicolumn{3}{c}{Partículas} & Spin & Carga eléctrica \\

	\hline

	\multirow{3}{*}{Quarks} & $(u,d)_{L}$ & $(c,s)_{L}$ & $(t,b)_{L}$ & $(\frac{1}{2},\frac{1}{2})$ & $(\frac{2}{3},-\frac{1}{3})$ \\

							& $u_{R}$ & $c_{R}$ & $t_{R}$ & $\frac{1}{2}$ & $\frac{2}{3}$ \\

							& $d_{R}$ & $s_{R}$ & $b_{R}$ & $\frac{1}{2}$ & $-\frac{1}{3}$ \\

	\hline

	\multirow{2}{*}{Leptones} 	& $(\nu_{e},e^{-})_{L}$ & $(\nu_{\mu},\mu^{-})_{L}$ & $(\nu_{\tau},\tau^{-})_{L}$ & $(\frac{1}{2},\frac{1}{2})$ & $(0,-1)$ \\

								& $e_{R}^{-}$ & $\mu_{R}^{-}$ & $\tau_{R}^{-}$ & $\frac{1}{2}$ & $-1$ \\

	\hline

	\multirow{2}{*}{Bosones de Gauge} 	& \multicolumn{3}{c |}{g} & $1$ & $0$ \\

										& \multicolumn{3}{c |}{$W^{\pm}$, Z} & $1$ & $\pm1, 0$ \\

	\hline

	Bosones escalares & \multicolumn{3}{c |}{H} & 0 & 0 \\

	\hline

\end{tabular}
\end{threeparttable}
\label{smparticles}
\end{table}
\renewcommand{\arraystretch}{1}

En resumen, el SM tiene 19 parámetros libres: las 9 masas de los fermiones (considerando que los neutrinos tienen masa nula), las 3 constantes de acoplamiento de las interacciones, los 3 ángulos de mezcla de la matriz CKM junto con la fase de de la violación CP, el ángulo de vacío de QCD y finalmente la masa del Higgs y su valor de expectación del vacio.


\subsection{Física mas allá del Modelo Estándar}

El SM provee una descripción notablemente exitosa de todos los fenómenos accesibles con los experimentos de altas energías disponibles actualmente. Sin embargo, también se sabe que el SM deja cuestiones sin resolver, tanto desde el punto de vista teórico, como experimental.

El triunfo de la teoría electrodébil, parece indicar que todas las interacciones corresponden a distintas manifestaciones de un único campo unificado y que el SM es una teoría efectiva a bajas energías (del orden de los 100 GeV).  Incluso ante la ausencia de la gran unificación de las fuerzas electrodébil y fuerte a una escala muy alta de energía, el SM debería ser modificado para incorporar los efectos de la gravedad a la escala de Planck.

Otro síntoma de incompletitud es la gran cantidad de parámetros libres (19) que deben ajustarse a los datos observados, ya que no resultan de principios teóricos más fundamentales. Más aun, el SM no explica por qué el cociente entre la escala de la interacción electrodébil y la escala de Planck es tan chico (<<problema de jerarquía>>), ni por qué la masa del Higgs es mucho ($10^{17}$)más liviana que la masa de Planck (<<problema de naturalidad>>).

Desde el punto de vista experimental, también existen algunos resultados que no pueden acomodarse dentro del SM. Distintos experimentros demostraron que si bien los neutrinos tienen una masa muy pequeña, la misma no es nula. En contraposición con el SM que considera a los mismos no masivos. De todas formas, es posible escribir un término de masa para los neutrinos en el lagrangiano, pero el mismo requiere de la existencia de neutrinos con quiralidad izquierda, que aún no fueron observados.

El SM tampoco provee un candidato para la materia oscura. A partir de la observación del movimiento de las galaxias, se sabe que el mismo no se corresponde con la cantidad de materia observada, y es por eso que se propone la existencia de materia indetectable para los instrumentos astronómicos de medición actuales. La materia oscura debería corresponder entonces a partículas masivas, que interactúen solo débilmente y gravitacionalmente.

Es por esto que existen diferentes teorías que proponen soluciones a estos problemas del SM. Una de ellas, y la que concierne a esta tesis, es la Supersimetría.

\section{Supersimetría}

Como se mencionó anteriormente, el SM ha tenido un gran éxito en la descripción de los fenomenos conocidos hasta la escala del TeV. Aun así, es clara la necesidad de construir una nueva teoría que solucione los problemas que el SM conlleva. El principal incovneniente es solucionar el <<problema de jerarquía>>, en el cual el cociente de escalas $M_{W}/M_{P}$ es muy pequeño. Para ello veamos lo que produce esta diferencia de escalas.

La parte eléctricamente neutra del campo de Higs del SM es un escalar complejo $H$ con un potencial clásico $V=m_{H}^{2}|H|^{2}+\lambda|H|^{4}$. El SM necesita un valor de exprectación de vacío (VEV) para H no nulo, en el mínimo del potencial. Esto curre si $\lambda >0$ y $m_{H}^{2}<0$, resultando en $\left\langle H \right\rangle = \sqrt{-m_{H}^{2}/2\lambda}$. Experimentalmente, de las edidas de las propiedades de las interacciones débiles, se sabe que el valor de $\left\langle H \right\rangle$ es de aproximadamente 174 GeV. El descubirmiento del bosón de Higgs en el 2012 con una masa cercana a 125 GeV implica que, suponiendo que el SM es correcto como una teoría efectiva, $\lambda = 0.126$ y $m_{H}^{2}=-(92.9 GeV)^{2}$.

Por cada partícula a la que se acopla el campo de Higgs, $m_{H}^{2}$ recibe una gran corrección cuántica de los efectos virtuales. Por ejemplo, si el campo de Higgs se acopla a un fermión $f$ con un término en el lagrangiano igual a $-\lambda H\hat{f}f$, el diagrama de Feynman en la figura (...) genera una corrección:

\begin{equation}
\Delta m_{H}^{2}=-\frac{|\lambda_{f}|^{2}}{8\pi^{2}}\Lambda_{UV}^{2}+...
\label{fermion_corr}
\end{equation}

donde $m_{f}$ es la masa del fermión y $\Lambda_{UV}$ es el corte usado para regular la integral en el loop. 

Si $\Lambda_{UV}$ es del orden de $M_{P}$, la corrección a $m_{H}^{2}$ es 30 órdenes de magnitud más grande que el valor requerido $\sim (100 \egev)^{2}$, ... Si bien los fermiones y bosones de gauge no tienen este comportamiento cuadrático en las correcciones cuánticas, tabién se ven afectados indirectamente por este efecto, ya que las masas de los mismos dependen de $\left\langle H \right\rangle$. De esta forma, todas las masas de SM se ven afectadas por la escala de corte $\Lambda_{UV}$.

Una forma de solucionar este problema consiste en considerar la existencia de un escalar complejo $S$ de masa $m_{S}$, que se acopla al campo de Higgs con un término $-\lambda |H|^{2}|S|^{2}$. El diagrama de Feynman de la figura (...) genera una corrección :

\begin{equation}
\Delta m_{H}^{2}=-\frac{|\lambda_{S}|^{2}}{16\pi^{2}}\left[\Lambda_{UV}^{2}-2m_{S}^{2}\ln(\Lambda_{UV}^{2}/m_{S})+...\right]
\label{boson_corr}
\end{equation}

De esta forma, si existiera un simetría que relacione fermiones y bosones, las contribuciones a las masas de las ecuaciones \ref{fermion_corr} y \ref{boson_corr} se cancelarían. A esta simetría se la denomina supersimetría (SUSY, por sus siglas en inglés) \cite{Martin:1997ns}.

Una transformación supersimétrica convierte un estado bosónico en uno fermiónico, y viceversa. El operador Q que genera estas transformaciones debe ser un espinor anticonmutativo, con:

\begin{equation}
Q boson= fermion
\end{equation}

Los espinores son intrínsecamente objetos complejos, por lo tanto el conjugado hermítico de $Q$ es también un generador de la simetría. Debido a que $Q$ y $Q^{\dagger}$ son operadores fermiónicos, llevan momento angular de espín $\frac{1}{2}$, por lo tanto es claro que SUSY debe ser una simetría espacio-temporal y los operadores $Q$ y $Q^{\dagger}$ deben satisfacer un álgebra de la siguiente forma:

\begin{equation}
\{Q,Q^{\dagger}\}=P^{\mu}
\end{equation}

\begin{equation}
\{Q,Q\}=\{Q^{\dagger},Q^{\dagger}\}=0
\end{equation}

\begin{equation}
[P^{\mu},Q]=[P^{\mu},Q^{\dagger}]=0
\end{equation}

donde $P^{\mu}$ es el momento generador de las traslaciones espacio-temporales.

Los estados de partícula de una teoría supersimétrica son representados en el álgebra de SUSY como supermultipletes. Cada supermultiplete contiene ambos estados, fermión y bosón, que son comúnmente llamados supercompañeros uno de otro.

Los generadores $Q$ y $Q^{\dagger}$ conmutan con los generadores de las transformaciones de gauge, por lo tanto las partículas en un mismo supermultiplete tienen que estar en la misma representación del grupo de gauge, y tener la misma carga eléctrica, isoespín y color. Y como el operador de masa $-P^{2}$ también conmuta con los generadores y con todos los operadores de rotación y traslación, deberán tener los mismos autovalores de $-P^{2}$ , y entonces la misma masa.

Cada supermultiplete tiene que contener igual número de grados de libertad fermiónico que bsónico ($n_{F}=n_{B}$), por lo que existen varias combinaciones posibles. Las dos más importantes para esta teoría son el supermultiplete quiral (o escalar) y el de gauge (o vectorial).

El supermultiplete escalar tiene un único fermión de Weyl ($n_{F}=2$) y dos escalares reales ($n_{B}=1$). Estos dos escalares se suelen poner como un único campo escalar complejo.

El supermultplete vectorial contiene un bosón vectorial de spin $1$. Para que la teoría sea renormalizable, tiene que ser un bosón de gauge no masivo, al menos antes de que la simetría de gauge sea espontáneamente rota. En este caso, este bosón contiene dos estados de helicidad ($n_{B}=2$). Por lo tanto su supercompañero es un fermión de Weyl de espín $\frac{1}{2}$, con dos estados de helicidad ($n_{F}=2$). Si en vez de esto, se intenta usar un fermión de espín $\frac{3}{2}$ la teoría no sería renormalizable. Los bosones de gauge deben transformar como la representación adjunta del grupo de gauge, por lo que sus compañeros fermiónicos, llamados <<gauginos>>, también.En el caso de incluir la gravedad, el gravitón de espín 2 (con dos estados de helicidad, n B = 2) tiene un supercompañero de espín llamado <<gravitino>>.



\subsection{Modelo Estándar Supersimétrico Mínimo}

En una extensión supersimétrica del SM, cada una de las partículas elementales conocidas está contnida en un supermultiplete quiral o de gauge, y debe tener un supercompañero con espín que difiera en 1/2 . La extensión que requiere la introducción de la mínima cantidad de partículas se conoce como <<Modelo Estándar Supersimétrico Mínimo>> (MSSM por sus siglas en inglés).

Veamos ahora como se van construyendo los distintos supermutipletes. Como los supermultiplestes escalares son los únicos que pueden contener un fermion cuya parte izquierda y derecha transforman de forma diferente, todos los fermiones del SM estan agrupados en este tipo de supermultiplete. Los nombres de los compañeros de espín 0 de los quarks o leptones son construidos anteponiendo una “s” (de scalar ), y son llamados <<squarks>> y <<sleptones>>. La parte izquierda y derecha de los quarks y leptones son fermiones de Weyl con diferentes propiedades de transformación de gauge del SM, entonces cada uno debe tener un compañero escalar complejo. Por ejemplo, los supercompañeros de la parte izquierda y derecha del campo de Dirac de los electrones son llamadas $\tilde{e}_{L}$ y $\tilde{e}_{R}$ , aunque el subíndice no se refiere a la helicidad de los slectrones (ya que ambos tienen espín 0) sino a la de sus supercompañeros. Lo mismo aplica para las demás leptones y quarks, los neutrino son simplemente denominados $\tilde{\nu}$ ya que son siempre izquierdos.

El bosón escalar de Higgs debe estar en un supermultiplete quiral ya que tiene espín 0. Dada la naturaleza de los campos quirales introducidos en la  mplementación de SUSY, el campo escalar de Higgs no es suficiente para dar masa a los fermiones de helicidad izquierda y derecha, por lo que se debe agregar un nuevo campo escalar para compensar. En el SM, el campo de Higgs es un doblete, y de los cuatro grados de libertad solo uno permanece como consecuencia de la ruptura de la simetría electrodébil, resultando en un bosón de Higgs. En el MSSM se necesitan dos dobletes de Higgs, $(H_{u}=(H^{+}_{u},H^{0}_{u})$
y $(H_{d}=(H^{0}_{d},H^{-}_{d})$. El escalar neutro que corresponde al bosón de Higgs del SM es una combinación lineal de $H^{0}_{u}$ y $H^{0}_{d}$. La nomenclatura usual para referirse a los supercompañeros de espín  es agregar “-ino” a la partícula del SM, por lo tanto los compañeros fermiónicos de los escalares de Higgs son denominados <<higgsinos>>, y se denotan con $\widetilde{H_{u}}$ y $\widetilde{H_{d}}$.

Los bosones vectoriales del SM tienen que estar en supermultipletes de gauge y sus supercompañeros fermiónicos son llamados <<gauginos>>. Las interacciones de gauge de color de QCD son mediadas por el gluón, cuyo compañero supersimétrico de espín  es el <<gluino>>. Los gauginos supercompañeros de los bosones de gauge electrodébiles, luego de mezclarse con los supercompañeros de los bosones de Higgs, dan lugar a los autoestados de masa denominados <<charginos>> y <<neutralinos>>. En la tabla \ref{susyparticles} se puede ver el espectro completo del MSSM.


\renewcommand{\arraystretch}{1.3}
\begin{table}	
\centering
\begin{threeparttable}
\caption{Supermultipletes quirales y de \textit{gauge} del MSSM.}
\begin{tabular}{ P{4cm} P{4cm} P{4cm} }

	\hline

	Supermultiplete & Bosón & Fermión \\

	\hline

	gluón, gluíno & $g$ & $\widetilde{g}$ \\

	\hline

	W, wino & $W^{\pm}$, $W^{0}$ & $\widetilde{W}^{\pm}$, $\widetilde{W}^{0}$ \\
	B, bino & B & $\widetilde{B}$ \\

	\hline

	\multirow{2}{*}{sleptón, leptón $^{*}$} 	& $(\widetilde{\nu},\widetilde{e})_{L}$ & $(\nu,e)_{L}$ \\

										& $e_{R}$ & $e_{R}$ \\

	\hline

	\multirow{3}{*}{squark, quark $^{*}$}		& $(\widetilde{u}_{L},\widetilde{d}_{L})$ & $(u_{L},d_{L})$ \\

										& $\widetilde{u}_{R}$ & $u_{R}$ \\

										& $\widetilde{d}_{R}$ & $d_{R}$ \\

	\hline

	\multirow{2}{*}{Higgs, higgsinos}	& $(H^{0}_{d},H^{-}_{d})$ & $(\widetilde{H}^{0}_{d},\widetilde{H}^{-}_{d})$ \\

										& $(H^{+}_{u},H^{0}_{u})$ & $(\widetilde{H}^{+}_{u},\widetilde{H}^{0}_{u})$ \\

	\hline

\end{tabular}
\begin{tablenotes}
\item [*] \footnotesize Junto con las otras dos generaciones.
\end{tablenotes}
\end{threeparttable}
\label{susyparticles}
\end{table}
\renewcommand{\arraystretch}{1}


Cada partículas y su supercompañero debe tener la misma masa, por lo que deberían existir, por ejemplo, fotinos de masa nula y selectrones con $0.511 \emev$. Como ninguna de las partículas antes mencionada fue observada experimentalmente, se deduce que SUSY es una simetría que está rota en el estado de vacío elegido por la naturaleza.

El hecho de que sea una simetría rota, impide que se cancelen las divergencias cuadráticas en el cuadrado de las masas escalares, y eso fue uno de los motivos por el cual se introdujo SUSY. Para poder garantizar que siga ocurriendo esa cancelación, la ruptura de la simetría debe ser suave, y el lagrangiano efectivo del MSSM tiene que escribirse como:

\begin{equation}
\mathcal{L}=\mathcal{L}_{SUSY}+\mathcal{L}_{soft}
\end{equation}

donde $\mathcal{L}_{SUSY}$ contiene todas las interacciones de gauge y de Yukawa, y preserva la invarianza supersimétrica. El conjunto de parámetros que aparecen en el lagrangiano $\mathcal{L}_{SUSY}$ son:

\begin{itemize}

	\item las constantes de acoplamiento de gauge $g_{s}$, $g$ y $g'$ correspondientes a los grupos de gauge $SU(3)_{C}$, $SU(2)_{L}$ y $U(1)_{Y}$, respectivamente

	\item los acoplamientos de Yukawa que describen las interacciones entre fermiones y bosones de Higgs

	\item el parámetro de masa del campo de Higgs $\mu$.

\end{itemize}


El lagrangiano que rompe SUSY, $\mathcal{L}_{SUSY}$, no está completamente determinado y su forma explícita así como el conjunto de parámetros inv lucrados dependen del mecanismo particular de ruptura de SUSY. Debido a que este mecanismo es desconocido, se puede suponer un conjunto de términos de
ruptura de la forma más general posible, sin indagar en sus orígenes, que se fijan solo pidiendo la invarianza frente $SU(3)_{C}\times SU(2)_{L}\times U(1)_{Y}$, y que sean suaves a fin de mantener la cancelación de las divergencias cuadráticas. Estos términos soft proveen exitosamente las masas de las partículas supersimétricas, a fin de que sean más pesadas que sus correspondientes compañeras del SM, y la ruptura espontánea de la simetría electrodébil requerida a bajas energías necesaria para explicar la generación de las masas de las partículas del SM. Aun así, la diferencia de masa entre supercompañeros no debe ser demasiado grande, ya que se perdería la solución al problema de jerarquía.

% El conjunto de nuevos parámetros que aparece en el superpotencial de ruptura de SUSY incluye \cite{Agashe:2014kda,Haber:1993wf}:

% \begin{itemize}
 
% 	\item Las masas soft del sector de Higgs, $m_{1}$ , $m_{2}$ y $m_{12}$ , donde $m_{12}^{2} \equiv B\mu$, $\mu$ es el parámetro introducido anteriormente y $B$ es el parámetro de ruptura de SUSY.

% 	\item Las masas soft de los squarks y sleptones de cada generación: $m_{\tilde{Q}}$ , $m_{\tilde{U}}$ , $m_{\tilde{D}}$ , $m_{\tilde{L}}$ y $m_{\tilde{E}}$ .

% 	\item Los acoplamientos trilineales de squarks y sleptones $A_{q}$ y $A_{l}$.

% 	\item Las masas soft de los gauginos, $M_{3}$ , $M_{2}$ y $M_{1}$ , asociadas a los grupos de gauge del $SU(3)_{C}$, $SU(2)_{L}$ y $U(1)_{Y}$, respectivamente.

% \end{itemize}

El análisis de la referencia \cite{Dimopoulos:1995ju} muestra que el MSSM posee 124 parámetros independientes. De estos, 18 corresponden a los parámetros del SM, uno corresponde al sector de Higgs (el análogo a la masa del Higgs del SM), y 105 son nuevos parámetros del modelo.




\section{Colisión \textit{pp}}

El LHC es un colisionador de protones, por lo tanto para comprender los procesos que ocurren en el mismo, es necesario entender la estructura del protón. Su composición se puede describir mediante la cromodinámica cuántica (QCD) \cite{ellis1996}, que explica las interacciones entre partículas que poseen carga de color: quarks y gluones. Los mediadores de la interacción, los gluones, pueden interactuar consigmo mismo, lo que produce que la fuerza dependa de la distancia entre las cargas. De esta forma, la constante de acoplamiento de la fuerza, aumenta a grandes distancias (o bajas energías) y disminuye para distancias menores (altas energías). Es por este motivo que los cálculos perturvativos solo se pueden efectuar a altas energías. Otra característica de la interacción es el confinamiento, es decir, que las partículas con color no puedan existir libremente. Solo estados de color neutro de múltiples partículas de color pueden ser observados en la naturaleza viajando distancias macroscópicas.

El proton es un barión, constituído por dos quarks \textit{u} y un quark \textit{d}, cada uno con una carga de color tal que deje al protón en un estado neutro. Estos tres quarks son llamados quarks de valencia del protón, y están rodeados por un mar de gluones y pares de quark-antiquark que surgen de fluctuaciones cuánticas. A altas energías la colision entre protones se puede considerar como una colisión entre dos de sus constituyentes, aplicando el <<modelo de partones>>.Este modelo fue introducido por Feynman \cite{PhysRevLett.23.1415} y Bjorken \cite{PhysRev.185.1975} a fines de los años 60, para interpretar los resultados de los experimentos de dispersión inelástica profunda (DIS) electrón-nucleón en SLAC. Los quarks de valencia y los quarks y antiquarks del mar junto con los gluones son llamados <<partones>> del protón. Cada partón lleva solo una fracción del momento y la energía del protón. Para la medición de una sección eficaz de dispersión fuerte que involucre quarks y gluones en el estado inicial, es necesario conocer el momento de las partículas incidentes. Como los partones solo llevan una fracción del momento del protón, y están en interacción permanente entre ellos, el momento es desconocido, por lo que la escala de energía de las colisiones varía. Además, como se mencionó, los quarks (\textit{q}) y gluones (\textit{g}) salientes no pueden observarse directamente debido al confinamiento, pero son observados en el detector como jets. Entonces no es posible medir una sección eficaz partónica como $\sigma(qg \rightarrow qg)$, pero se puede hacer una medida inclusiva, como la sección eficaz hadrónica $\sigma(pp \rightarrow jj)$ con dos jets en el estado final. En teoría de perturbaciones, para pasar desde la sección eficaz partónica a la sección eficaz hadrónica es necesario conocer la probabilidad de que un partón de tipo \textit{n} sea encontrado con una fracción de momento \textit{x}, es decir, las funciones de distribución partónica (PDF). Estas funciones son determinadas a partir de datos obtenidos de los propios experimentos de altas energías, ya que no pueden determinarse a partir de la teoría. 

Esta conexión entre los hadrones observables y el nivel partónico es posible gracias al concepto de <<factorización>>, que permite una separación sistemática entre las interacciones de corta distancia (de los partones) y las interacciones de larga distancia (responsables del confinamiento de color y la formación de hadrones). El teorema de factorización \cite{ELLIS1978281} establece que la sección eficaz de producción de cualquier proceso de QCD del tipo $A + B \rightarrow X$, siendo $a_{i}$ ($b_{j}$) los constituyentes del hadrón inicial $A$($B$), puede ser expresada como: 

\begin{equation}
\sigma_{AB\rightarrow X}=\sum_{ij}\int dx_{a_{i}}dx_{b_{j}}f_{A/a_{i}}(x_{a_{i}},\mu_{F}^{2})f_{B/b_{j}}(x_{b_{j}},\mu_{F}^{2})\sigma_{a_{i}b_{j}}(\mu_{F}^{2},\mu_{R}^{2})
\end{equation}

donde $x_{i}$ ($x_{j}$) es la fracción del momento del hadrón $A$($B$) que lleva el partón $a_{i}$ ($b_{j}$) y $\sigma_{a_{i}b_{j}\rightarrow X}$ es la sección eficaz de la interacción a nivel partónico, calculada a un dado orden de perturbaciones y una escala de renormalización $\mu_R$ . La escala de renormalización es introducida para absorber las divergencias ultravioletas que aparecen en los cálculos perturbativos más allá del primer orden. Las funciones $f_{A/a_{i}}(x_{a_{i}},\mu_{F}^{2})$ son las PDF, que representan la probabilidad de encontrar un partón de tipo \textit{n} en el hadrón \textit{h} con una fracción de momento $x_{n}$ , dada una escala de factorización $\mu_{F}$. Esta escala es un parámetro arbitrario introducido para tratar singularidades que aparecen en el régimen no perturbativo. Estas divergencias son absorbidas, en forma similar a la renormalización, dentro de las funciones de distribución partónicas a la escala $\mu_{F}$. 


A modo de ejemplo, en la Figura \ref{cross_section}, se muestra el buen acuerdo entre la sección eficaz de algunos procesos del SM medidas por ATLAS y las predicciones teóricas. Las observaciones experimentales realizadas en LHC resultan compatibles con el SM a un nivel de muy alta precisión.


\begin{figure}[ht]
\centering
\includegraphics[width=1\textwidth]{cross_section}
\caption{Resumen de las distintas medidas de sección eficaz de producción de procesos del SM, comparadas con sus valores teóricos esperados.}
% \vspace{0.2cm}\footnotesize\textbf \sl{...}\vspace{0.2cm}
\label{cross_section}
\end{figure}
