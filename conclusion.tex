\chapter{Conclusiones}
% \addcontentsline{toc}{chapter}{Conclusión}
\chaptermark{Conclusiones}

Supersimetría se ha vuelto uno de los principales objetivos del LHC en los últimos años para la búsqueda de nueva física. Esa búsqueda consiste en encontrar eventos por encima de los ya predichos por el SM, que toman el rol de fondo en la búsqeda de SUSY. El hallazgo de un exceso en los datos sobre las predicciones del SM abrirán el camino para dar respuesta a los interrogantes actuales en física de partículas. En el caso de que los datos sean compatibles con las predicciones del SM pondrá nuevas cotas en los parámetros de teorías de nueva física, en particular de nuevas partículas predichas por SUSY. En ambos casos los resultados nos permitirán avanzar en nuestra comprensión de la naturaleza. 

En esta tesis se ha presentado un estudio de la producción de eventos de $W$/$Z$ + jets o $\ttbar$ , en procesos donde un electrón del estado final es reconstruido como un fotón, contaminando así la búsqueda de nuevas partículas en regiones de señal que contienen fotones, jets y energía faltante en el estado final. En particular, se implementó un método para poder estimar este tipo de procesos utilizando información de los datos colectados por el detector ATLAS durante el Run 2 del LCH en los años 2015 y 2016.

Los resultados obtenidos en esta tesis son compatibles con predicciones anteriores [cita]  y fueron validados en las denominadas regiones de control y validación. Se concluye además, una idependencia del método en los criterios de identificación de los electrones, puediendo ser estos tanto \textit{medium} como \textit{tight}.

Al presente, los resultados mostrados en este trabajo están siendo implementados por la colaboración ATLAS para estimar el conjunto total de fondos del MS en las regiones de señal en la búsqueda partículas Superimétricas con decaimiento y presencia de fotones en el estado final.
