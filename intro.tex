\chapter*{Introduction}
\addcontentsline{toc}{chapter}{Introduction}
\chaptermark{Introduction}

Quantum Chromodynamic (QCD) is theory of strong interaction. It is formulated in terms of elementary fields (quarks and gluons). For a suffient hevavy quark like the charm, bottom and top, the cross section is calculabel as a perturbation sireies in the QCD running coupling constant $\alpha_s$. 

In the context of heavy-ion physics, the beauty production may become a very important source of \jpsi\ mesons LHC and RICH energies. One of the signal of the formation of Quark Gluon Plasma (QGP) is the \jpsi\ production suppresion. 

In the context of LHC  as well for the CP violation studies in the B sector, but also in other studies where \bbbar\ events becomes a source of background. In this context better understaning of \bbbar\ production will improve the desing of trigger strategies. 

At present there are compleate calulations up to NLO in perturvation theory. But this predictions fails to give an accurate prediction on the \bbbar\ production since at near threshold higher order terms present large contibutions. this make \bbbar\ production at near threshold a suitable freame work to test new develomts at and thoretical toll to include higher order terms in the preturvative expantion, thus increasing our knowlege beyond NLO. 

\bbbar\ production has been measured twices in fix target experiments but they are incompatible as well as suffering from larges statistical errors. A more precision measuremnt will help to clarify the situation in this sector. Durign year 2000 partialy commisioned HERA-B detector a \bbbar\ cross section measurem was perfromed, but still it suffer from limited statistic. During the end of 2002 and begining of 2003 HERA-B took it last data, this time making it possible to increas the accuracy of previous meassurement deu to the incresed statistic. in this thesis we report on the new measuremt based in the 2002-2003 data sample.


{\bf Outline }

We first describe the theoretical framework  of the \bbbar\ production -perturbative QCD- describing the renormalisation procedure needed by the theory and the technics to include high order terms in the perturbative expansion. In chapter 2 we describe the HERA-B detector and setup for the data taking period of 2002-2003. In chapter 3 we describe the data taking conditions and data samples used in this thesis together we a description of the simulated data needed for the determination of detector and trigger efficiencies. 
In chapter 4 we presen a deeper description of the First Level Trigger system together with a study of it performance during 2002-2003 data taking period. 
In chapter 5 we describe the determination of the \bbbar\ cross section through the inclusive \btojpsi\ + X decay . Finally in the last chapter we describe the search of exclusive fully reconstructed decays in the \bjk\ and \bjkp\ decay channels.  